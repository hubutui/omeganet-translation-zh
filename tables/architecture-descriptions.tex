\renewcommand{\captiontitle}{\CNN{} architecture variants considered}
\begin{table*}
\begin{center}
\begin{tabular}{lrrrrr} \hline
\toprule
Name      & \UNet{} 0 & \UNet{} 1 & \UNet{} 2 & \UNet{} 3 & Millions of Parameters \\
\midrule
Network A & 128       & --        & --        & --        & \MillionsOfParamsA{} \\
Network B & 64        & 64        & --        & --        & \MillionsOfParamsB{} \\
Network C & 64        & 64        & 64        & --        & \MillionsOfParamsC{} \\
Network D & 64        & 64        & 64        & 64        & \MillionsOfParamsD{} \\
\bottomrule
\end{tabular}
\caption[\captiontitle{}]{
\captiontitle{}.
\hl{
Network A is the baseline \UNet{} (the \hl{initial} segmentation module alone, without transformation or \hl{final} segmentation modules).
Networks B, C, and D are full \omeganet{} architectures with 1, 2, and 3 \UNet{} components, respectively, in the fine-grained segmentation module.
\UNet{} 0 is the \UNet{} in the \hl{initial} segmentation module; \UNet{}s 1, 2, and 3 are the first, second, and third \UNet{} components in the \hl{final} segmentation module, as applicable.
For each \UNet{} component of each network variant, the length of the feature vector is provided.
}
}
\label{tab:architecture-descriptions}
\end{center}
\end{table*}
