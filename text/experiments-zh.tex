
\subsection{\hl{HCMNet 数据集}}

\renewcommand{\captiontitle}{考虑中的 \CNN{} 网络结构变体}
\begin{table*}
\begin{center}
\begin{tabular}{lrrrrr} \hline
\toprule
名称      & \UNet{} 0 & \UNet{} 1 & \UNet{} 2 & \UNet{} 3 & 参数量(百万) \\
\midrule
网络 A & 128       & --        & --        & --        & \MillionsOfParamsA{} \\
网络 B & 64        & 64        & --        & --        & \MillionsOfParamsB{} \\
网络 C & 64        & 64        & 64        & --        & \MillionsOfParamsC{} \\
网络 D & 64        & 64        & 64        & 64        & \MillionsOfParamsD{} \\
\bottomrule
\end{tabular}
\caption[\captiontitle{}]{
\captiontitle{}.
\hl{
网络 A 是一个基线 \UNet{}(仅有粗分割模块,不含变换或者细分割模块).
网络 B、C 和 D 是细分割模块中分别有着 1、2和3个 \UNet{} 结构的完整的 \omeganet{} 网络结构.
表中还给出了每一个 \UNet{} 模块的特征向量长度.
}
}
\label{tab:architecture-descriptions}
\end{center}
\end{table*}


HCMNet 数据集包含 \NumPtT{} 个样本,其中 \NumPtO{} 个患有肥厚性心肌病,\NumPtC{} 为健康人 \citep{Ho2017}.
心脏核磁共振 \CMR{} 检查由十个临床中心于 2009 到 2011 年期间按照标准流程进行.
其中九个使用 1.5-T 核磁共振,一个使用 3-T 核磁共振.
我们采集了三个 \SA{} 切面,一个 \HLA{} 切面和一个 \VLA{} 切面的 \SSFP{} 序列.
\hl{
图像的平面间隔为 $\SpacingMU{} \pm \SpacingSD{}$mm,尺寸为 $\MatrixMU{} \pm \MatrixSD{}$ 像素;更加详细的信息可以参考文献 \citet{Ho2017} 的补充材料.
}

左室心肌和其他所有四个心腔都在 \SA{},\HLA{} 和 \VLA{} 切面进行手动分割标注(应当注意到,并不是所有的类别都可以在 \SA{} 和 \VLA{} 切面看到).
\hl{
\ND{2} 时间卷数据会送入 ITK-Snap \citep{Yushkevich2006} 中去;每隔五帧做一个手动的分割标注,而剩余的图像的标注由插值算法自动产生.(分割由本文第一作者完成,他有有着五年的手动分割标注 \CMR{} 的经验).
}
\LV{} 和 \RV{} 切面的心肌不包含乳头肌和小梁.

每一卷数据数据都被适当地裁剪或者补零到 $\N \times \N$ 的空间尺寸,而时间轴的范围为 $\NumFramesMin{}$ 到 $\NumFramesMax{}$ 帧.
不均匀的背景光照均做了修正,然后进行直方图均衡处理.
每一个单独的图像都做了归一化处理,也就是减去均值再除以标准差,然后在送入 \CNN{} 网络.

\subsubsection{训练与交叉验证}

至于交叉验证,所有的被试分成了三份($\NumImFoldA{}$,$\NumImFoldB{}$,$\NumImFoldC{}$ 张图),并确保属于一个被试者的图像都分到了同一组中.
每一个网络(见表 \ref{tab:architecture-descriptions})都在其中两组中训练,然后在剩下那组进行测试,且包括所有的组合.
\hl{
网络 A 仅有粗分割模块;考虑到 \UNet{} 在生物医学图像分割任务中表现不错,这就是个基线水平.
网络 B、C 和 D 分别为细分割模块使用 1、2 和 3 个 \UNet{} 模块的 \omeganet{} 网络.
}

网络使用正交化的权重 \hl{\citep{Saxe2013}} 初始化,并使用 Adam 优化器 \citep{Kingma2015} 优化.
学习率初始化为 $0.001$,并每隔 $26$ 代衰减为原来的 $0.1$.
为了避免过拟合,我们使用数据增强(平移和缩放 $\pm \AugTrans{}\%$;旋转 $\pm \AugRot{}$\degree),并设置粗分割模块的权重衰减系数为 $\weightdecay{}$.
值得注意的是,数据增强也\emph{隐含}在最后的细分割模块,因为前面用了变换网络.我们还对每一时间序列做了单独的数据增强.

\subsubsection{性能评价指标}

我们计算每一图像的预测值与真实值之间的加权的 \IoU{} 来衡量网络性能.
对于二值图像(只有一个前景和一个背景)来说,真实值 $I_T$ 和预测值 $I_P$ 的 \IoU{}(也称 Jaccard 系数)定义为:

\begin{equation}
\IoU{} \left( I_T, I_P \right) = \frac{|I_T \cap I_P|}{|I_T \cup I_P|},
\end{equation}

\noindent 注意,在代码实现中应给分母加上一个较小的数字,以避免除以零的情况.
为了将 \IoU{} 扩展到多类的情况,我们先分别计算每一类与背景的 \IoU{}.
然后计算一个加权的 \IoU{},其中的权值由这些类别占所有类别的比重决定,也就是求出了一个加权的均值前景 \IoU{}.

\subsubsection{实现}

这个模型使用 Tensorflow \hl{\citep{Chollet2015,Abadi2016}} 的 Keras 接口,用 Python 语言实现,并使用一块 12 GB 显存的英伟达 Titan X 图形处理器(GPU)训练.
所有的网络在训练中都需要大约 20 分钟迭代一次.
而在测试阶段,网络预测出分割结果的速度大约在 15 帧每秒.

\hl{
\subsection{2017 MICCAI \miccaidata{} 数据集}

网络 B 从头开始在 2017 MICCAI \miccaidata{} 数据集上训练.
训练数据包含 100 个患者(20 个正常,20 个心肌梗塞,20 个扩张型心肌病 20 个肥厚性心肌病和 20 个右室心脏病)的 \SA{} 切面数据.

所有切片的数据都提供了舒张末期和收缩末期左室心肌,左室血池和右室血池的真实分割标签
为了能与 \miccaidata{} 的结果进行对比,网络的性能评估同时使用了 \IoU{} 和 Dice 系数:

\begin{equation}
\mathrm{Dice} \left( I_T, I_P \right) = \frac{2|I_T \cap I_P|}{|I_T| + |I_P|},
\end{equation}

\noindent 根据目前最优的结果 \citep{Isensee2017,Isensee2018},网络使用了五折交叉验证来训练.
}
