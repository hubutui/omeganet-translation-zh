本文中,我们提出了 \omeganet{}:一个全新的深度卷积神经网络结构,可以用于定位、方向对齐和分割.
我们将该网络应用到了全自动的全心分割,同时将图像变换到一个典型方向,从而大大地降低了 \SSFP{} \CMR{} 图像的分析难度.
无需先验知识,我们从头开始端到端的训练了该网络,用来对三个切面(\SA{}、\HLA{}和\VLA{})的五个前景类别(心脏的四个心腔加上左室心肌)进行分割.
从解剖的角度来看,这个数据集是高度异质的,包含了健康的受试者和肥厚性心肌病患者.
数据采集自10个医疗中心,包括 1.5-T 和 3-T 核磁共振图像.
在交叉验证实验中,网络在变换参数预测与分割任务中均表现良好.

\hl{
\omeganet{} 也在 2017 MICCAI \miccaidata{} 公开数据集的两个类别中的分割任务中取得最好的成绩.
相比于我们内部的 HCMNet 数据集,\miccaidata{} 包含了更多的 \LV{} 和 \RV{} 病例,但是只有一个临床切面,和更少的前景类别.
而且,HCMNet 数据集是多中心研究的结果,而\miccaidata{} 则来自单中心.

之前最好的分割 \citep{Isensee2017,Isensee2018} 同时集成了 2D 和 3D \UNet{} 结构,为堆叠的视频序列进行优化.
他们的方法也因此不适用于 \HLA{} 和 \VLA{} 切面,因为这些切面通常只采集一个静态图.
因此,我们的 \omeganet{} 不仅分割性能比 \citet{Isensee2018} 好,而且也更加通用,同时还能提供 \citep{Isensee2017} 所不具备的定位和方向对齐信息.
}

本研究在这么四个方面具有创新性.
第一,这个网络预测了三个切面中的五个前景类别,如 \citep{Vigneault2017} 所言,这个任务是更加困难和棘手的.
第二,我们用一个空间变换网络模块 \citep{Jaderberg2015} 来将每个切面都变换到一个典型方向去.
\hl{
\CNN{} 既不是旋转不变的也不是尺度不变的.
从技术上来说,我们是可以通过获取一个足够大的数据集,里面包含了各种可能的旋转变化,从而克服这个缺点.
但是,生物医学图像的获取和标注都是十分昂贵和耗时的,这也就催生了我们的这个设计.
通过将最后的细分割的输入图像进行标准化的变换,我们简化了网络的设计,同时还模拟了医师的操作.
}
第三,这个网络结构启发自 \citet{Viola2001} 的级联分类模型,我们使用 \UNet{} $0$ 来进行粗分割(为了预测变换参数),然后变换后的图像被送入细分割模块(\UNet{}s $1$,$2$和$3$).
\hl{
最后,\omeganet{} 从其本身的设计出发,提供了一定的可解释性,包括中间层的输出(粗分割结果和变换参数)和最后的细分割.
这样子,我们的网络比起 \UNet{} 结构提升了复杂度和可预测的信息,但是无寻担心 CNN 无法解释的黑盒子问题.
}

尽管该数据集包含了三个切面,也包含了健康受试者和有左心疾病的患者,我们还是可以将该数据集扩充到其他更加通用的场景.
首先,其他常用的心脏切面(例如轴面、三腔和右室长轴面)应该在未来的工作中添加到数据集中来.
另外,在其他的 \CMR{} 图像序列(例如灰度超声心动图)或者模态(比如心脏 CT 和超声心动图)的测试也很值得尝试.
而且,将这个网络应用到其他也需要定位、重新对齐方向和分割的生物医学图像任务中也是很有意思的,比如胎儿影像.
\hl{
最后,我们希望 \omeganet{} 能在其他需要多个临床切面分割的应用中也能有效应用,例如 \CMR{} 运动校正和切面配准 \citep{Sinclair2017}.
}

\omeganet{} 还有很多可以优化的地方.
例如,我们可以训练网络对单帧图像进行分割,而无需空间或者时间的上下文信息;修改网络结构以共享时间序列和空间切片的信息,从而提升网络性能.
E-Net(``Efficient Net'')提供了一个改进版的 \UNet{} 模块,它能有效地提升计算效率和降低显存使用,同时还能保证准确率 \citep{Paszke2016};这些经验已经有效地应用到了心脏的分割 \citep{Lieman-Sifry2017},理论上来说应该也能应用到这里.
