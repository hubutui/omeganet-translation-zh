In this work, we have presented the \omeganet{}: a novel deep convolutional neural network (\CNN{}) architecture for \hl{localization}, orientation alignment, and segmentation.
We have applied this network to the task of fully automatic whole-heart segmentation and simultaneous transformation into the ``canonical'' clinical view, 
which has the potential to greatly simplify downstream analyses of \SSFP{} \CMR{} images.
The network was trained end-to-end from scratch to segment five foreground classes (the four cardiac chambers plus the \LV{} myocardium) in three views (\SA{}, \HLA{}, 
and \VLA{}), without providing prior knowledge of the view being segmented.
The dataset was highly heterogeneous from the standpoint of anatomical variation, including both healthy subjects and patients with overt hypertrophic cardiomyopathy.
Data was acquired from both 1.5-T and 3-T magnets as part of a multicenter trial involving 10 institutions.
In cross-validation experiments, the network performed well in predicting both the parameters of the transformation, and the cardiac segmentation.

\hl{
\omeganet{} also achieved state-of-the-art performance on the publicly available 2017 MICCAI \miccaidata{} dataset in two of three classes.
Compared with our internal HCMNet dataset, \miccaidata{} contains a broader range of \LV{} and \RV{} pathologies, but only one clinical view, and fewer foreground classes.
Moreover, HCMNet was a multicenter study, whereas \miccaidata{} was acquired at a single center.
It is encouraging that \omeganet{} performed well on both datasets.

The prior state-of-the-art \citep{Isensee2017,Isensee2018} was achieved using an ensemble of 2D and 3D \UNet{}-inspired architectures, optimized for \emph{stacked} cine series.
Their method is therefore not generally applicable to \HLA{} and \VLA{} views, which are typically acquired as single slices.
Therefore, \omeganet{} outperformed \citet{Isensee2018} while remaining more general, and while providing \hl{localization} and orientation information not predicted by \citep{Isensee2017}.
}

The work is novel in \hl{four} principal ways.
First, this network predicts five foreground classes in three clinical views, which is a substantially more difficult problem than has been addressed previously in the literature \citep{Vigneault2017}.
Second, a spatial transformer network module \citep{Jaderberg2015} was used to rotate each view into a canonical orientation.
\hl{
\CNN{}s are neither rotation invariant nor equivariant, nor scale invariant.
From a technical standpoint, in theory this shortcoming can be addressed by acquiring very large datasets which adequately represent all possible rotations.
However, biomedical imaging datasets are expensive and time consuming both to acquire and to annotate, directly motivating this design decision.
By standardizing the orientation of the input to the \hl{final} segmentation module, we simplify the task of both the downstream network and the physician interpreting the images.}
Third, the proposed architecture takes loose inspiration from the cascaded classifier models proposed by \citet{Viola2001}, in that \UNet{} $0$ performs \hl{initial} segmentation (in order to predict transformation parameters), and the transformed image is then provided as input to a \hl{final} segmentation module (\UNet{}s $1$, $2$, and $3$).
\hl{
Last, by its design, \omeganet{} provides human-interpretable, intermediate outputs (an \hl{initial} segmentation and transformation parameters) in addition to the \hl{final} segmentation.
In doing so, we substantially increase the complexity and information predicted by the network compared to the \UNet{} architecture, but without adding concerns that \CNN{}s are ``black boxes'' whose internals cannot be adequately interrogated.
}

Although the dataset included three orthogonal cardiac planes and both healthy \hl{subjects and those with \LV{} pathology}, there remain potential opportunities to extend the dataset to more general scenarios.
First, other cardiac planes used in clinical practice (such as the axial, three-chamber, and \RV{} long axis views) should be added in future work.
It would also be useful and interesting to test this on other \CMR{} pulse sequences (such as gradient echo) and on additional modalities (i.e., cardiac computed tomography and echocardiography).
Moreover, it could also be interesting to apply this technique to other areas within biomedical image segmentation where \hl{localization}, reorientation, and segmentation are useful, such as in fetal imaging.
\hl{
Finally, we expect \omeganet{} to be useful in applications requiring the segmentation of multiple clinical planes, such as \CMR{} motion correction and slice alignment \citep{Sinclair2017}.
}

A variety of opportunities present themselves in terms of optimizing the \omeganet{} architecture.
For example, the network was trained to segment individual image frames, without spatial or temporal context; modifying the architecture to allow information sharing between temporal frames and spatial slices has the potential to increase accuracy and consistency.
The E-Net (``Efficient Net'') provides modifications to the \UNet{} blocks which increase computational and memory efficiency, while preserving accuracy \citep{Paszke2016}; these lessons have been applied successfully to cardiac segmentation \citep{Lieman-Sifry2017}, and could theoretically be applied here as well.
